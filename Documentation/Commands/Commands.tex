\documentclass[letterpaper,titlepage]{article}
\begin{document}
	\title{BayoutBot Instruction Reference}
	\author{Brandon Oubre}
	\date{\today}
	\maketitle
	
	\newcommand{\instrDesc}[2]{
		\textbf{#1} \\
		#2 \\
	}

	\section*{AVR Pin Numbering}
	\begin{tabular}{l | c | l}
	\textbf{AVR Pin} & \textbf{BayouBot Pin} & \textbf{Function} \\
	D0 (2) & - & Reserved for Rx with Bluetooth \\
	D1 (3) & - & Reserved for Tx with Bluetooth \\
	D2 (4) & 0 & GPIO \\
	D3 (5) & 1 & GPIO \\
	D4 (6) & 2 & GPIO \\
	B6 (9) & 3 & GPIO \\
	B7 (10) & 4 & GPIO \\
	D5 (11) & 5 & GPIO \\
	D6 (12) & 6 & GPIO \\
	D7 (13) & 7 & GPIO \\
	B0 (14) & 8 & GPIO \\
	B1 (15) & 9 & GPIO \\
	B2 (16) & 10 & GPIO \\
	B3 (17) & - & Unused \\
	B4 (18) & - & Unused \\
	B5 (19) & - & Unused \\
	C0 (23) & 11 & GPIO \\
	C1 (24) & 12 & GPIO \\
	C2 (25) & 13 & Reserved for Motor Control \\
	C3 (26) & 14 & Reserved for Motor Control \\
	C4 (27) & 15 & Reserved for Motor Control \\
	C5 (28) & 16 & Reserved for Motor Control \\
	\end{tabular}
	\clearpage

	\section*{Instruction Table}
	\begin{tabular}{l | c | l | c}
	\textbf{Command} & \textbf{Byte} & \textbf{Arguments} & \textbf{Length} \\
	Set INPUT & 0x01 & The pin to set as an input & 2 \\
	Set OUTPUT & 0x02 & The pin to set as an output & 2\\
	HIGH & 0x03 & The pin to set HIGH & 2 \\
	LOW & 0x04 & The pin to set LOW & 2 \\
	READ & 0x05 & The pin to read. & 2 \\
	\end{tabular} \\
	\instrDesc{Set INPUT}{The BayouBot will assign the argument pin as an input pin.  Consult the pin numbering table for a list of acceptable BayoutBot GPIO pins.}	
	\clearpage
\end{document}
